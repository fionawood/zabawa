\section{Discussion}

\subsection{Major data implications}
Our data indicates that although users evaluate badges as being reasonably motivating regardless of task type, badges tangibly affect performance only in the “highly engaging” (text adventure) task scenario. This may imply that although users are attracted to the specific reinforcement and extra interactivity that badges may generate, their potency is highly dependent on the types of in-task events that they are hinged on. For example, a badge achieved for unscrambling a word may not provide an effective cue beyond the actual act of the word being unscrambled, but a badge achieved for unlocking a stage in the text adventure adds significance to the event that cannot already be derived from the game mechanics.

A result that has even broader potential for future research is that enjoyment was negatively correlated with engagement regardless of incentive type. Although this does not necessarily imply that there is a general inverse relationship between the two factors, it does motivate investigation into the task variables that may cause an a change in one but not the other. 

\subsection{Experimental limitations}
Although several limitations of our experimental design restrict the amount and diversity of meta-analysis that it is possible to conduct on the current data set, there are several alterations that could be made in the future to elucidate some of the more nuanced effects that the experiment could have produced. For example, although the influence of game and incentive order (such as having the games in ascending or descending order of average engagement) could theoretically be analyzed, the way that game records are stored for each user makes this line of investigation impractical. A subsequent investigation might benefit from adding “ordering effects” to the list of independent variables that are manipulated between sessions. 

Another minor change to the experiment that could have made our results more robust would have been to track the “motivational baseline” for each task type by presenting participants in the “none” incentive condition with a question analogous to “how motivating did you find [incentive]?” Although our challenge and enjoyment survey questions, as well as engagement time under the “none” conditions may serve as reasonable proxies for this value, the fact that no direct comparison exists for this question in the incentive and non-incentive conditions is a flaw in our experimental design.

Also in the vein of retooling experimental design, a far more drastic alteration would be to track the user’s actual actions – including key presses, clicks, and idling – for patterns that correlate with engagement, enjoyment, and challenge (or lack thereof). This would allow for a more granular analysis of what the actual inputs and symptoms of engagement are, and how they might be observationally different from the signs of pure enjoyment, frustration, or confusion.

\subsection{Future directions}

Our finding that enjoyment and engagement were decoupled begs the question of which factor is most important for the successful completion of different types of tasks – for example, increasing enjoyment may lead to better results in an imaging tagging task, whereas engagement may do the same for text transcription. It stands to reason that a variety of variables, including task complexity, expertise, and familiarity may change which factor leads to better outcomes.

Furthermore, it would also be useful to investigate whether incentive schemes that don’t affect engagement do impact other aspects of gameplay – for example, efficiency, accuracy, speed, and time spent idling. The fact that users reported that they did find several of the incentives to be motivating in situations where this did not translate into engagement gains indicates that there might be some as-of-yet unexplored factor into which this additional motivation is channeled.