\section{Related Work}

\subsection{Incentive schemes and motivational research in the psychological sciences}
Traditionally, scientific investigation into incentives and task motivation has been restricted to cognitive psychology research. Psychologists in the seventies and eighties conducting now-seminal studies were surprised to find that external motivators such as monetary gain reduced subjects’ engagement in tasks they once found intrinsically interesting, a phenomenon now referred to as the “overjustification effect”~\cite{lepper1973undermining}. Modern theories have still failed to achieve consensus about the exact psychological underpinnings behind this seemingly paradoxical reversal, although prominent explanations involve self-perception theory (in which an individual infers his or her motivations from external events), cognitive dissonance (in which one’s motivations change on the basis of actions taken), and framing (the alteration of perceptions of identical situations on the basis of differing contexts).

A majority of these forays into incentive schemes in the psychological sciences has involved the use of financial rewards as motivators, limiting their direct applicability to our research, which attempts to tease out the effectiveness of entirely different incentive schemes. However, psychological literature has found that even a dissimilar application of financial rewards can lead to surprisingly different outcomes. For example, a classical study found that being paid a disproportionately small amount for a task actually increased engagement in the activity~\cite{rosenfield1980rewards}. The implications behind this caveat have been investigated in modern studies that attempt to explain this phenomenon in terms of inequity of task input and task output~\cite{Ryan00}. This idea of inequity could represent a useful operationalization for incentive potency in our own work.

\subsection{Applications of psychological research on incentives to computational systems}
A particularly salient review of the applications of the psychological conception of incentives and motivation to computer systems can be found in Finneran and Zhang \cite{Finneran05}, who approach the issue from the perspective of maximizing user “flow” within computer-mediated environments. Their work represents an exhaustive coverage of potential motivators and task manipulations that could enhance engagement with computer systems, as well as an exploration of the ways that flow has been defined and operationalized in both psychological and computer science contexts. Furthermore, these authors propose a model for flow in computer-mediated environments that takes into account the effect of the intervening medium of the computer system that they refer to as “Person, Artifact, Task” (PAT) model of flow. 
	Direct applications of the theories discussed above to computational systems have primarily been the domain of research into reinforcement learning (RL) for artificial intelligence systems~\cite{barto2004intrinsically, kiili2005digital,barto2005intrinsic}. 
One of the more transferable applications of computer science research into “flow” and motivation is in the idea that the “skill” of a user must roughly correspond to the challenge represented by the task~\cite{barto2004intrinsically,barto2005intrinsic}, which derives from investigations into the best ways to apply gamification techniques and design game systems for the purpose of serving educational tasks. Adaptive adjustment of a task’s difficulty level to the user’s existing level of ability optimizes flow, enjoyment, and intrinsic motivation to continue an activity, something which is highly relevant to keep in mind for our current research.

\subsection{Gamification mechanism design}
Existing research gives us insight into the effects of individual motivational strategies on games and other tasks, but rarely are different incentives compared with one another or contrasted with the user's existing interest in the task subject. Examination of the effects of competition and competitiveness shows that users are affected differently by motivation schemes depending on their personalities~\cite{Song13}; more competitive users are driven by competition in games, but less competitive users are discouraged by it. One might extrapolate that similar effects could be produced by taking advantage of other elements of gameplay in ways that might enhance one player's experience while detracting from another's. Easley et. al.~\cite{Easley13} compared competitive badge schemes (in which a badge is earned by being better than other users) with absolute schemes (in which a badge is earned through achieving a fixed set of accomplishments) through the lens of the contributor-reward model rather than the contribution-reward model. We aim to extend this user-centric approach to a broader examination of motivation and reward to provide perspective among different motivation schemes rather than optimizing within only one type.