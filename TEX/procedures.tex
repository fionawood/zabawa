\section{Procedures}

\subsection{Project introduction and participant consent}
On entering the project, subjects see a short introductory page that explains that our three tasks intend to explore how language is latently processed in situations that it is either made salient or not made salient. This ensures that rather than hyper-focusing on the changing incentive schemes and task types, participants are going through the series of tasks with as natural a mindset as possible. The fact that each of our tasks is somewhat linguistically related (a part-of-speech matching game, a word scramble, and a text adventure) makes this cover story seem reasonably plausible and should reduce potential suspicion of behavioral manipulation. A consent form follows the introduction, the terms of which each participant must agree to to be proceed to the experiment.

\subsection{Task and incentive assignment}
Tasks and incentives are assigned on a pseudo-random basis, with each possible combination of task and incentive pre-defined and assigned sequentially to the next participant to ensure equal participation among all permutations.

Each task type can be associated with any four of our incentive schemes: social comparison, leveling up, badges, and “none”. Participants will be given all three of the task types in a single testing session in the aforementioned pseudo-random order. Each of these tasks will have a different incentive scheme assigned to it out of the four available to ensure that none of the participants become aware of the study’s true purpose and to prevent direct comparisons within individual sessions.

After an 8000 millisecond interval, a “next” button will slowly appear on the layout. Clicking on this button will direct the user to the game’s survey page. The user may continue his current task for as long as he wishes, and the time interval for which he does so after the appearance of the next button will be recorded as one of our indicators of task engagement. 

\subsection{Task types and structure}
{\bf Text Adventure}. The “text adventure” task consists of a module styled similar to classic text-adventure games of the early 1990’s. The user may input commands based on the previous prompts, and actions that are allowed within the context of the game result in advancement of the story. Certain points in the adventure result in branches, in which the course of the story is altered depending on the user’s decisions. This task was hypothesized to result in the greatest amount of engagement due to the cognitive effort required to focus on both the story and the possibilities for future commands.

{\bf Word Scramble}. The “word scramble” features a bouncing sphere containing a scrambled English word. The user is prompted to type guesses for the unscrambled word into the provided input box. When a word is unscrambled, it is replaced with another scrambled word. After a certain amount of successes, the length of scrambled words increases to make the game more challenging. This task was hypothesized to generate medium levels of engagement.

{\bf Word Bounce}. The final, “word bounce”, task consists of a multitude of bouncing spheres with various English words in their centers. The user is prompted to type in all of the nouns that they can identify in the spheres. Once a noun is found, it is highlighted in red on the appropriate sphere. This task was hypothesized to generate the lowest levels of engagement due to its self-limiting and cognitively untiring nature.

\subsection{Incentive types and structure}
{\bf Badges}. The badge incentive features a graphical badge with a descriptive label (such as “Escapee”) appearing at the bottom of the user’s screen. Badges persist throughout the game, accumulating sequentially at the bottom of the screen as achievements are triggered. This incentive provides information about objective progress in the task, but does not explicitly inform the user of their distance from the endgame. It is hypothesized to generate the most direct relationship to the behavior desired from the user, as it ties specific rewards to specific actions.
{\bf Levels}. The level 
{\bf Leaders (i.e. “social comparison)}. The leader, or social comparison, incentive consisted of text informing the current user what percentage of other players they are currently better than. This incentive does not provide any information about the user’s absolute progress in the task, but is hypothesized to generate motivation or engagement through providing information about relative performance compared to other players. The actual percentage values of this incentive are fixed, anchored to the same points within gameplay as badges and levels.


\subsection{Likert scales and preference testing}

A brief survey will appear after every type that asks the user for his opinions on the task’s difficulty, his feelings of personal efficacy in doing the task, his interest in the task itself, the task’s potential for replayability, whether he noticed the incentive scheme, and whether he felt that the incentive scheme was motivating, informative, and/or increased their engagement.

At the end of all three games, the user will be asked to rank both the tasks with their incentive schemes and the incentive schemes alone in order of preference.

\subsection{Debriefing}
On completing all of their assigned tasks, participants will be directed to a closing page with a debrief explaining the true purpose of the experiment. They will be provided with the option to receive the results of the genuine experiment once all data is processed and conclusions are finalized.
