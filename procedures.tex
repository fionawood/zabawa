\section{Procedures}
 \subsection{Initial Deception}
On entering the project, subjects will see a short introductory page that explains that our three tasks intend to explore how language is latently processed in situations that it is either made salient or not made salient. This deception ensures that rather than hyperfocusing on the changing incentive schemes and task types, participants are going through the series of tasks with as natural a mindset as possible. The fact that each of our tasks is somewhat linguistically related (a part-of-speech matching game, a word scramble, and a text adventure) makes this cover story seem reasonably plausible and should reduce potential suspicion of deception.
 
 \subsection{Task and Incentive Assignment}
Tasks and Incentives are assigned on a pseudorandom basis, with each possible combination of task and incentive pre-defined and assigned sequentially to the next participant to ensure equal participation among all permutations.

Each task type can be associated with any four of our incentive schemes: social comparison, leveling up, badges, and “none”. Participants will be given all three of the task types in a single testing session in the aforementioned pseudorandom order. Each of these tasks will have a different incentive scheme assigned to it out of the four available to ensure that none of the participants become aware of the study’s true purpose and to prevent direct comparisons within individual sessions.

After a two-minute interval, a “next” button will appear on the layout that will direct the user to the task questionnaire. The user may continue his current task for as long as he wishes, and the time interval for which he does so after the appearance of the next button will be recorded as one of our indicators of task engagement.  
 \subsection{Likert Scales and Preference Testing}

A brief survey will appear after every type that asks the user for his opinions on the task’s difficulty, his feelings of personal efficacy in doing the task, his interest in the task itself, the task’s potential for replayability, whether he noticed the incentive scheme, and whether he felt that the incentive scheme was motivating, informative, and/or increased their engagement. 

At the end of all three games, the user will be asked to rank both the tasks with their incentive schemes and the incentive schemes alone in order of preference.

 \subsection{Debriefing}
On completing all of their assigned tasks, participants will be directed to a closing page with a debrief explaining the true purpose of the experiment. They will be provided with the option to receive the results of the genuine experiment once all data is processed and conclusions are finalized.
